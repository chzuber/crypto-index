\documentclass[11pt]{article}

\usepackage{amsmath}
\usepackage{amssymb}
\usepackage{bbm}
\usepackage{booktabs}
\usepackage{natbib}
\usepackage{color}
\usepackage{caption}
\newcommand\fnote[1]{\captionsetup{font=scriptsize}\caption*{\textsl{Note:} #1}}
\newcommand\fnotes[1]{\captionsetup{font=scriptsize}\caption*{\textsl{Notes:} #1}}
\usepackage{rotating}
\usepackage{hyperref}

\title{LCI 20}
\author{Onno Kleen \and Christopher Zuber}
\begin{document}
\maketitle
\section{Outline}

Our index Lykke Crypto Index 20 (LCI 20) is a weighted average of current market capitalization.

Some Text.\footnote{Code and .tex-files can be found at \href{https://github.com/onnokleen/crypto-index}{https://github.com/onnokleen/crypto-index}}
 
We focus on weighted average but have maximum weights

\cite{Trimborn2016}

Perfect for ETFs: An equal-quantity portfolio with $q_{i,t} = c$, $c \in \mathbb{R}$, is a one-to-one replication of our LCI (20). Perfect for building ETFs - no rebalancing costs.

\subsection{Definition}

Let $t$ denote time index starting at time $t_0$, $t = t_0, t_1, t_2, \dots$. And let $\mathcal{C}_t$ denote the set of coins that are traded at time $t$.

\begin{itemize}
  \item Price $p_{i,t}$: price of asset $i$ at time $t$
  \item Quantity $q_{i,t}$: overall number of shares/items per asset $i$ at time $t$
  \item Market Capitalization of asset $i$ at time $t$ is $c_{i,t}$
\end{itemize}

$$ \text{LCI20}_t = \frac{\sum_{i \in \mathcal{C}_{t}} w_{i,t} c_{i,t}}{Divisor} $$

One disadvantage of the real-time re-weighting is that 

\subsection{Addressing splits}\label{subseq:split_smoothing}
For anticipated splits/forks we propose an adaptive smoothing technique for addressing ``insane'' price movements in direct aftermath. 

\subsection{Calculating LCI 20}

\begin{enumerate}
  \item Calculate each coins market share $s_{i,t}$.
  \item Truncate market shares by maximum $\bar s$: $\bar s_{i,t} = \max\{ s_{i,t}, \bar s\}$
  \item Rescale them, so weights sum up to one: $w_{i,t} = \frac{\bar s_{i,t}}{\sum_{i \in \mathcal{C}_t} \bar s_{i,t}}$. Note that $w_{i,t}$ is afterwards greater than $0.25$ due to rescaling.
  \item $$\widetilde{\text{LCI20}}_t = \sum_{i \in \mathcal{C}_{t}} w_{i,t} c_{i,t}$$
  \item The initial value of the weighted sum is given by $$\widetilde{\text{LCI20}}_{t_0} = \sum_{i \in \mathcal{C}_{t}} w_{i,t_0} c_{i,t_0}$$
  \item $Divisor = \widetilde{\text{LCI20}}_{t_0} * 100$
\end{enumerate}

\subsection{Questions to address}
\begin{itemize}
  \item Why 20 currencies? 19-09-2017 14:41 20th market capitalization (STEEM) is only \$286.382.955 and 24 hour trading volume of \$686. Further, the 20 currencies with the highest market capitalization have a share of 92 \% of the total market capitalization and a 24 hour trading volume of 94 \% of the total trading volume of September 22, 2017. If we add another 30 currencies, the 24 hour trading volume increases by 2 percentage points where we think that a smaller but better traceable index outweighs an index with a larger but more volatile currency base.
  \item ``Dead coins'' a problem?
  \item If there is a split (like Bitcoin), new currency is part of Lykke 20 but is part of constituents-check at the end of the week.
  \item Basis: 100 Punkte?
  \item How to get market capitalization of public float?
  \item Maximal weight maybe 20\%? DAX: Maximum weight 10\%.
\end{itemize}

Wikipedia:  In general, the large holdings of founding shareholders, corporate cross-holdings, and government holdings in partially privatized companies are excluded when calculating the size of a public float.

https://www.coindesk.com/rethinking-bitcoin-market-cap/: In 2014, NVIDIA engineer John Ratcliff theorized that approximately 30\% of the current bitcoin supply is made up of ``zombie'' bitcoins that have been inactive for more than a year This number includes bitcoins connected to inaccessible wallets, government-seized bitcoins, ``burned'' bitcoins and bitcoins abandoned during the early days of bitcoins, including Nakamoto's mythical stash of over a million bitcoins.

\subsection{How does it work in other indices}

\begin{itemize}
  \item DAX: Weighting based on market capitalization of public float (bedeutet Streubesitz, keine Aktien von Langzeitanlegern  wie Familie Porsche/Quant).
\end{itemize}

\subsection{Features}

\begin{itemize}
  \item
  \item Constituent changes each week. Maybe Friday? Maybe based on trade volume in last 7 days? Good against ``dead coins''.
\end{itemize}

\begin{table*}
\caption{Table explaining differences: proposal vs.\ example}
\centering
\resizebox{\textwidth}{!}{
\begin{tabular}{ccc}
  Issue & Our Proposal & Example \\
  \midrule
  Frequency & Real-time & Daily \\
  Split-smoothing & See Subsection \ref{subseq:split_smoothing} & Not necessary due to daily data\\
  Public float & Only use coins that have been traded the last 4 years - weekly updated (real-time tracking difficult) & nothing \\
  ``Zombie coins'' & Only use coins that have been traded the last 4 years & only use 0.30\%\\
\bottomrule
\end{tabular}
}
\fnotes{In this table differences in implementation of our example and our actual proposal are reported.}
\end{table*}






\section{Example}

Data is from \href{https://www.kaggle.com/sudalairajkumar/cryptocurrencypricehistory}{https://www.kaggle.com/sudalairajkumar/cryptocurrencypricehistory}

or CoinCap.io via Rest API

Something nice to illustrate:
\begin{itemize}
  \item Volatility in August (Bitcoin-split versus July). Show new composition after split.
\end{itemize}

\section{Splits}

\bibliographystyle{ecca}
\bibliography{crypto-index.bib}

\end{document}
